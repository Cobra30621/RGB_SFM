\documentclass[class=NCU_thesis, crop=false]{standalone}
\begin{document}

\chapter{總結}
\section{結論}

本研究提出了一種基於卷積神經網路的新型可解釋性深度學習模型。
此模型以C.F Yang 在 2022 年提出的CIM模型\cite{YangCNNInterpretable}為基礎,
將彩色影像分為顏色和輪廓兩個完全不同的方面,
各自輸入進色彩感知區塊與輪廓感知區塊,
進行彩色卷積和灰階卷積以提取色彩與輪廓特徵。
之後將這些特徵分別輸入顏色特徵傳遞區塊和輪廓特徵傳遞區塊,
透過模擬皮層的多層架構將特徵的空間性進行時序合併並傳遞到下一層,
最終利用全連接層學習影像的分類特徵形成分類結果。

從實驗結果來看,
我們可以知道此模型的每一層均可以進行可視化的呈現,
隨著越後面的層數其可視化出來的特徵變越趨近於完整,
並且透過這些可視化的圖片來對模型的決策過程進行解釋。
此外,我們也可以這些可視化結果來分析模型或資料集上的不足並且加以改進,
進而提升模型的效果與可解釋性。

本研究的重要意義除了提高了CIM模型的性能和泛用性外,
在色彩的特徵提取與可視化方法上更是提出了創新的方法,
透過引入PCCS色相環和6種基礎色的概念,
設計了色彩提取區塊,解決了分別處理顏色和輪廓的衝突問題,
有效的提取出單純的顏色特徵,
使得模型專注於學習顏色特徵成為可能。
這些改進有助使未來彩色可解釋性領域的研究可以更進一步。

\section{未來展望}
本研究尚有一些能夠改進的地方,希望能在未來繼續改善和加強,
使得未來的研究可以在此模型的基礎上進行更進一步的探索與優化,
以解決現實中真實存在的問題。

目前在輪廓特徵傳遞層我們利用歐式距離並輸入高斯函數計算濾波器與輸入的相似度,
使得濾波器可以學習到輸入影像的特徵。
然而歐式距離對特徵的位移、旋轉、縮放均缺乏穩定性,
使得在處理更複雜的資料集往往需要更多的濾波器數目來進行處理,
未來可以嘗試尋找其他可以取代歐式距離的方法,
進一步使得模型的濾波器可以學習到更有效的特徵。

此外,也可以更進一步研究如何將這種模擬人類視覺與大腦結構的模型,
應用在現實中的實際問題,
使得此類模型可以真正的幫助到現實中的人們。
這些努力將有助於推動可解釋模型領域的發展,
使得可解釋性模型獲得更進一步的發展與創新。

\end{document}