\documentclass[class=NCU_thesis, crop=false]{standalone}
\begin{document}

\chapter{總結}
\section{結論}

本研究提出了一種基於卷積神經網路的新型可解釋性深度學習模型。
將彩色影像分為顏色和輪廓兩個完全不同的方面進行學習,
各自輸入進色彩感知區塊與輪廓感知區塊,
進行彩色卷積和高斯卷積以提取色彩與輪廓特徵。
之後將這些特徵分別輸入顏色特徵傳遞區塊和輪廓特徵傳遞區塊,
透過模擬皮層的多層架構將特徵的空間性進行時序合併並傳遞到下一層,
最終利用全連接層學習影像的分類特徵形成分類結果。

從實驗結果來看,
我們可以知道此模型的每一層均可以進行可視化的呈現,
隨著越後面的層數其可視化出來的特徵變越趨近於完整,
並且透過這些可視化的圖片來對模型的決策過程進行解釋。
此外,我們也可以這些可視化結果來分析模型或資料集上的不足並且加以改進,
進而提升模型的效果與可解釋性。

本研究的重要意義除了提高了CIM模型的性能和泛用性外,
在色彩的特徵提取與可視化方法上更是提出了創新的方法,
透過引入PCCS色相環和6種基礎色的概念,
設計了色彩提取區塊,解決了分別處理顏色和輪廓的衝突問題,
有效的提取出單純的顏色特徵,
使得模型專注於學習顏色特徵成為可能。
這些改進有助使未來彩色可解釋性領域的研究可以更進一步。

\subsection{目前模型的限制}
本研究在實驗的過程中也發現了以下幾個限制。
第一點為處理現實世界影像的挑戰,我們在 CIFAR-10 資料集上的實驗結果顯示,模型在處理現實世界影像時,準確度有顯著下降。我們發現這主要是由於高斯卷積模組對於特徵變換(如旋轉、平移)需要大量不同的濾波器來學習,這在複雜影像中導致了濾波器數量的爆炸性增長。這一發現提示我們需要探索更有效的特徵學習方法,可能包括引入更適合處理複雜影像的卷積技術或特徵提取方法。
第二點為模型深度受限於影像解析度我們發現,模型深度的增加受到影像解析度的限制,這是由於每層模型必須加上空間合併模組。因此,當影像尺寸不足以支撐更深的模型時,模型的表現會受到影響。為了應對這一限制,我們在研究中嘗試了不同的影像尺寸和模型深度配置,並努力尋找最佳的平衡點。儘管如此,這仍然是一個需要進一步研究和改進的方向。
這些限制為我們提供了改進模型設計的重要線索。

\section{未來展望}
本研究尚有一些能夠改進的地方,希望能在未來繼續改善和加強,
使得未來的研究可以在此模型的基礎上進行更進一步的探索與優化,
以解決現實中真實存在的問題。

目前在輪廓特徵傳遞層我們利用歐氏距離並輸入高斯函數計算濾波器與輸入的相似度,
使得濾波器可以學習到輸入影像的特徵。
然而歐氏距離對特徵的位移、旋轉、縮放均缺乏穩定性,
使得在處理更複雜的資料集往往需要更多的濾波器數目來進行處理,
未來可以嘗試尋找其他可以取代歐氏距離的方法,
進一步使得模型的濾波器可以學習到更有效的特徵。

此外,也可以更進一步研究如何將這種模擬人類視覺與大腦結構的模型,
應用在現實中的實際問題,
使得此類模型可以真正的幫助到現實中的人們。
這些努力將有助於推動可解釋模型領域的發展,
使得可解釋性模型獲得更進一步的發展與創新。

\end{document}