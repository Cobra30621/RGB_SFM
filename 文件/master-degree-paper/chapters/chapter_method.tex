\documentclass[class=NCU_thesis, crop=false]{standalone}

\begin{document}

\chapter{研究方法}

\section{以卷積神經網路為基礎的RGB彩色可解釋性模型}
\subsection{模型架構}

此章節將介紹本論文所提出的可解釋性模型整體架構與每個部分的功能,並說明資料在模型中的運作方式,模型架構圖如\cref{fig:model_arch}。
整個模型可以分成三個部分,色彩感知區塊、輪廓感知區塊和特徵傳遞區塊。

\fig[1][fig:model_arch][H]{model_arch.png}[模型架構圖][模型架構圖]


色彩感知區塊透過彩色卷積模組計算輸入影像不同區塊的色彩與30種基本色彩的相似度去分析輸入影像的色彩分布,以提取影像中每個區塊的色彩特徵。
輪廓感知區塊透過將影像進行灰階化後使用高斯卷積層來提取影像中輪廓和邊緣的特徵。
特徵感知區塊使用了 CIM 模型模擬了大腦皮質的運作模式並對其進行優化,
每一層都會將底層的特徵資訊整合並傳遞到下一層進行學習。 
特徵傳遞區塊每層可以分成三個部分,
高斯卷積模組,特徵增強模組,空間合併模組,
高斯卷積模組負責學習與提取輸入的特徵,
特徵增強模組負責過濾不重要的特徵,空間合併模組則模擬皮層的資訊合併,融合眼球跳動的概念,將輸入的資訊根據空間位置關係進行合併。
此外,為了在訓練之後使用者可以對彩色和灰階的可解釋性圖片進行對照,
因此我們對模型加入了一個限制,
限制色彩特徵傳遞區塊和輪廓特徵傳遞區塊的層數和空間合併方式必須相同。
\pagebreak
\subsection{模型符號說明}
	由於本研究在可解釋性部分採用了CIM中的特徵圖解析方法來解析本論文模型的決策過程,
	因此在各個區塊我們均會產生特徵映射圖(Feature Map, FM)、特徵映射響應圖(Response Image, RM)和特徵映射圖之對應影像(Corresponding Image, CI)作為可解釋性的核心要素。
	具體的產生過程將會在\cref{section:InterablePicture}去介紹。

	\begin{description}
		\item[]\textit{B}$_{in}$, \textit{H}$_{in}$, \textit{W}$_{in}$:輸入影像之長、寬

		\item[]\textit{FM}$^{color}_{0}$:色彩感知區塊之彩色卷積模組的\textit{FM}
		\item[]\textit{RM}$^{color}_{0}$:色彩感知區塊之彩色卷積模組的\textit{RM}
		\item[]\textit{CI}$^{color}_{0}$:色彩感知區塊之彩色卷積模組的\textit{CI}
		\item[]\textit{H}$^{color}_{0}$, {W}$^{color}_{0}$:色彩感知區塊之輸出的長、寬

		\item[]\textit{FM}$^{color}_{i}$:色彩特徵傳遞區塊第i層之高斯卷積模組的\textit{FM}, i = 1$\sim$L
		\item[]\textit{RM}$^{color}_{i}$:色彩特徵傳遞區塊第i層之高斯卷積模組的\textit{RM}, i = 1$\sim$L
		\item[]\textit{CI}$^{color}_{i}$:色彩特徵傳遞區塊第i層之高斯卷積模組的\textit{FM}$_{i}$的對應影像, i = 1$\sim$L
		\item[]\textit{C}$^{color}_{i}$, \textit{H}$^{color}_{i}$, \textit{W}$^{color}_{i}$:色彩特徵傳遞區塊第i層之輸出的通道數、長、寬, i = 1$\sim$L
		
		\item[]\textit{FM}$^{gray}_{0}$:輪廓感知區塊之高斯卷積模組的\textit{FM}
		\item[]\textit{RM}$^{gray}_{0}$:輪廓感知區塊之高斯卷積模組的\textit{RM}
		\item[]\textit{CI}$^{gray}_{0}$:輪廓感知區塊之高斯卷機模組的\textit{CI}$_{gray}$的對應影像
		\item[]\textit{C}$^{color}_{i}$, \textit{H}$^{gray}_{0}$, {W}$^{gray}_{0}$:輪廓感知區塊之輸出的通道數、長、寬

		\item[]\textit{FM}$^{gray}_{i}$:輪廓特徵傳遞區塊第i層之高斯卷積模組的\textit{FM}, i = 1$\sim$L
		\item[]\textit{RM}$^{gray}_{i}$:輪廓特徵傳遞區塊第i層之高斯卷積模組的\textit{RM}, i = 1$\sim$L
		\item[]\textit{CI}$^{gray}_{i}$:輪廓特徵傳遞區塊第i層之高斯卷積模組的\textit{CI}$_{i}$的對應影像, i = 1$\sim$L
		\item[]\textit{C}$^{color}_{i}$, \textit{H}$^{color}_{i}$, \textit{W}$^{color}_{i}$:輪廓特徵傳遞區塊第i層之輸出的通道數、長、寬, i = 1$\sim$L
	\end{description}
\subsection{演算法流程}
Step 1:決定整個模型的架構與參數,色彩特徵傳遞區塊與輪廓特徵傳遞區塊之層數均為L

Step 2:對輸入的彩色影像做灰階前處理產生對應的灰階影像,此時彩色影像

Step 3:將彩色影像(B$_{in}$, 3, H$_{in}$, W$_{in}$)和灰階影像(B$_{in}$, 1, H$_{in}$, W$_{in}$)分別輸入色彩提取層和輪廓提取層。各自的輸出為色彩輸出(B$_{in}$, 30, H$^{color}_{0}$, W$^{color}_{0}$),灰階輸出(B$_{in}$, C${_{gray}}$, H$^{gray}_{0}$, W$^{gray}_{0}$)

Step 4:將色彩輸出和灰階輸出分別輸入進各自的特徵傳遞層進行綜合特徵的學習與合併,
		最終會獲得色彩特徵(B$_{in}$, C$^{color}_{L}$, H$^{color}_{L}$, W$^{color}_{L}$), 灰階特徵(B$_{0}$, C$^{gray}_{L}$, H$^{gray}_{L}$, W$^{gray}_{L}$)

Step 5:將色彩特徵與灰階特徵攤平為(B$_{in}$, C$^{Linear}_{in}$)並輸入進全連接層學習分類特徵

Step 6:計算 loss value 並進行反向傳播

\pagebreak



\section{色彩提取區塊設計與實現}

此章節將說明本論文所提出的色彩提取區塊設計與實現,
該區塊主要是透過將濾波器(Filter)設定30個調色基礎顏色來作為該區塊的權重,
並使用高斯卷積模組計算出影像中不同區塊之色彩與filter的相似度,
後將結果送入特徵增強模組,最終形成影像的色彩特徵。
這樣的方法目的在於分析輸入影像的色彩在基礎顏色中的成分後,
透過卷積操作來去模擬人眼的眼球跳動,
從而重現人眼獲得色彩特徵的效果。
以下將針對濾波器之設計、彩色卷積模組,兩個部分進行詳細說明。

	\subsection{濾波器之設計}
	由於色彩提取區塊的輸入為彩色影像其輸入通道分別為紅、綠、 藍三色的通道,
	我們設定輸出的通道數為固定為30,
	並且將濾波器的形狀設為 $\left(30 , 3\right)$,以表示為30種不同RGB色彩。

	在30種不同色彩的選擇上,
	我們利用日本色彩研究所於1965年提出的PCCS(Pratical Color Co-ordinate System)中的24色相環中(如\cref{fig:PCCS}所示)的顏色,
	再加上紅、綠、藍、黑、白、灰六個基礎色組成。
	由於以上的30色均為基礎顏色(如\cref{fig:30Base}),
	因此我們希望透過計算不同區塊的色彩與這些基礎顏色的相似度來獲得區塊的色彩成分。

	\fig[0.5][fig:PCCS][H]{chapter3/PCCS.jpg}[PCCS體系的24色色相環][PCCS體系的24色色相環\cite{PCCScite}]

	\fig[1][fig:30Base][H]{chapter3/30BaseColor.png}[濾波器設定的基礎30色][濾波器設定的基礎30色]

	\subsection{彩色卷積模組}
	在傳統的卷積操作中,
	我們會在輸入影像上使用一個稱為Windows的滑動方塊,
	並且與不同的濾波器進行內積運算來計算輸出,
	這樣的輸出結果只能代表Windows與不同濾波器在平面上的投影而並沒有相似度的意義。
	然而我們在彩色提取區塊的濾波器視為不同的色彩,
	因此我們在計算Windows時會將透過將Windows內的所有像素的RGB值進行加總與平均,
	計算出該Windows的平均RGB值作為該Windows的代表色。
	之後我們計算Windows的代表色與濾波器中不同的顏色的色差,
	並且利用高斯函數取代內積運算將色差轉換為相似度,
	以相似度作為反應強度進行輸出。

	在色差的計算上,我們曾經考慮過下面三種不同的色差計算方法,分別如下:
	\begin{itemize}
	  \item [1)] 
	  	計算RGB色彩空間的歐幾里得距離(簡稱Euclidean)來代表顏色之間的差距,
	  	其公式如\cref{eq:eq-RGBcdist}。
	  	他是三種方法之中最為快速
	  	$\Delta$C代表兩個顏色的色差,
	  	$\Delta$R為兩個顏色R值的差距,
	  	$\Delta$G為兩個顏色G值的差距,
	  	$\Delta$B為兩個顏色B值的差距。
	  	\begin{equation}
	    \label{eq:eq-RGBcdist}
	    	\Delta C = \sqrt{(\Delta R)^2 + (\Delta G)^2 + (\Delta B)^2}
		\end{equation}

	  \item [2)]
	  	由CompuPhase公司提出的一種低成本的加權歐幾里得距離公式\cite{LABformula}(簡稱LAB Euclidean),
	  	它的權重由RGB中的紅色在色彩中的分量多少而決定。
	  	這種色差計算方法的好處是利用這個公式便可以用兩個顏色的RGB值計算出接近CIE L*u*v(CIEL*a*b)空間中的距離,
	  	同時這套公式也被使用在CompuPhase自己的產品中。
	  	CIELAB是由國際照明委員會提出的色彩空間,
	  	是目前描述人眼可見所有顏色最完整的色彩空間,
	  	因此在CIELAB的距離也會較RGB空間之距離更接近人眼所視。
	  	其公式如\cref{eq:eq-LAB},
	  	其中C$_{1}$代表第一個顏色,
	  	C$_{2}$代表第二個顏色。
	  	\begin{equation}
	    \label{eq:eq-LAB}
	    \begin{split}
	    	\overline{r} & = \frac{C_{1,R} + C_{2,R}}{2} \\
	    	\Delta R & = C_{1,R} - C_{2,R} \\
	    	\Delta G & = C_{1,G} - C_{2,G} \\
	    	\Delta B & = C_{1,B} - C_{2,B} \\
	    	\Delta C & = \sqrt{(2 + \frac{\overline{r}}{256}) * (\Delta R)^2 + 4 * (\Delta G)^2 + (2 + \frac{255 - \overline{r}}{256}) * (\Delta B)^2}
	    \end{split}
		\end{equation}

	  \item [3)]
	  	根據伽馬校正理論,人們對於每個顏色的亮度感知呈現不同的非線性曲線,因此對於不同的色彩的權重應該要是不同的,因此我們參考了一種常見計算色差的加權歐幾里得距離的公式(簡稱Weighted Euclidean),公式如\cref{eq:eq-weightcdist}。
	  	\begin{equation}
	    \label{eq:eq-weightcdist}
	    	\Delta C = \sqrt{2 * (\Delta R)^2 + 4 * (\Delta G)^2 + 3 * (\Delta B)^2}
		\end{equation}
		然而,在\cite{LABformula}中也有提到,根據伽馬校正的理論,或許我們應該對不同的顏色分布的影像設定不同的加權值才會比較符合人眼的視覺對顏色亮度的感知。
	\end{itemize}
	這三個色差的計算方法各有其優缺點,
	並且經過我們的實驗後發現三者在準確度和可解釋圖片的產生上十分相似,
	但考量到Euclidean方法會容易受到RGB值的細微不同而產生變化。
	而根據上面提到\cite{LABformula}中的說法,
	如果使用Weight Euclidean方法會需要因應不同的資料集去調整加權數值,會增加問題的複雜度。
	因此我們最終選擇了穩定性較高的LAB Euclidean方法來計算色彩感知區塊中輸入影像和30類基礎顏色的色差值,
	做為評估相似度的依據。

	我們在此利用高斯函數取代內積的原因在於高斯函數是以距離為基礎的函數,
	其計算結果能夠反映出距離的意義並且表達向量間的相似度,
	而相似度在後續的解釋性表現中幫助我們更好的理解模型的可解釋性圖片產生之原因。\\
	高斯函數(Gaussian function):\\
	  $x$ 為輸入值、$m$ 為中心點、$\sigma$ 為寬度參數
	  \begin{equation}
	      \label{eqn:rbf-gaussian-function}
	      \phi (x) = \exp \left( -\frac{\| x-m \|^2}{2\sigma ^2} \right) 
	  \end{equation}

\pagebreak
\section{輪廓感知區塊設計與實現}
	\subsection{前處理設計}
	在輪廓感知區塊中,
	為了使該區塊能夠專注於提取輪廓特徵而不需考慮色彩因素,
	我們會對輸入影像進行前處理。
	首先,我們會將彩色影像轉換為灰階影像,
	轉換公式如\cref{eq:eq-grayscale},
	Gray代表灰階值,R、G、B各代表紅、綠、藍三色通道的值。
	\begin{equation}
	    \label{eq:eq-grayscale}
	    Gray = 0.299 * R + 0.587 * G + 0.114 * B
	\end{equation}
	這種加權方法主要考慮了人眼對於不同色彩的敏感度,
	從而能夠較為完整保留彩色圖像中的細節,
	接著為了消除不同色彩導致的灰度值差異,
	使該區塊能夠專注於輪廓資訊而不考慮色彩資訊,
	我們接者進行了max-min正規化,
	使每張影像的灰階值統一道$0\sim1$之間。
	透過上面步驟的前處理,
	我們從而可以確保輪廓感知區塊能夠專注於提取輪廓的特徵。

	\fig[0.7][fig:SFN_update][H]{chapter3/grayscale.png}[灰階前處理示意圖][灰階前處理示意圖]

	\subsection{濾波器初始化}
	在這個模型之中,除了色彩感知區塊外,
	其餘區塊的高斯卷積模組的濾波器我們均是使用 Kaiming Uniform 的方式,
	根據 kaiming 論文\cite{DBLP:journals/corr/HeZR015}中的方式將每個值初始化範圍為$[-\sqrt{\frac{6}{\text{fan\_in}}}, \sqrt{\frac{6}{\text{fan\_in}}}]$,
	$\text{fan\_in}$ 為輸入通道數。

	我們在實作中選擇使用 Kaiming Uniform 的原因是因為 Kaiming Uniformm 相較於常用的 Uniform 初始化和 Xavier Uniform\cite{pmlr-v9-glorot10a} 初始化多考慮了整流線性單位函數(ReLU)的存在,
	Uniform 初始化的方式無法解決隨著神經網路的增加而導致梯度消失的問題,
	Xavier Uniform 為了解決梯度消失問題加入了 rescale 函數 $\frac{1}{\sqrt{\text{n}}}$但卻只適用於激活函數為線性函數的情況下,
	而 Kaiming Uniform 在解決梯度消失的問題時同時考慮了激活函數為非線性函數的情況並在\cite{DBLP:journals/corr/HeZR015}透過實驗證明了kaiming uniform 在神經網路在不影響準確度的同時更快收斂。
	由於我們的模型中在特徵增強模組中使用的非線性函數ReLU的變形去進行特徵增強,因此選擇了 Kaiming Uniform 來去進行後面的實驗。



\pagebreak
\section{特徵傳遞區塊之優化設計}
\label{chapter:chapter3.4}
特徵傳遞區塊以 CIM 為基礎,對 CIM 的原始區塊進行深入分析和改進並對其進行優化,
我們希望可以這些優化措施可以在可解釋性不變的同時,又可以對模型的效能和準確度進行提升,
此外我們還希望這些優化措施可以將一些原本人工去分析資料集後指定的參數變成可訓練參數,
進一步提高模型的自動化與適應能力。

	\subsection{高斯卷積模組優化設計}
	在高斯卷積模組中,
	CIM的原始實作方法修改了傳統CNN\cite{726791}中的卷積操作,
	將CNN卷積操作的內積轉換為放射狀積底函數的高斯函數,
	目的是希望以距離為基礎的高斯函數可以表達輸入與Filter之間的距離關係並將之視為相似度。

	在實作上CIM採用了逐個計算每個 Windows 和濾波器之間的歐式距離並將結果輸入進高斯函數得出相似度。
	我們改進了這個實作過程,利用了 GPU 的多核優勢,
	將每次卷積的所有位置的 Windows 取出來後平行放入 GPU 的多個核心,
	同時計算這些 Windows 與濾波器的歐式距離再輸入高斯函數中得出相似度,
	這樣的實作方式使得整體模型的訓練速度可以大幅提高,
	也完整利用GPU的效能,提升了模型效率。

	\colorbox {yellow}{可以加入原本的卷積方式和平行處理的方式的示意圖}

	\subsection{特徵增強模組之優化設計}
	在特徵增強模組中,
	CIM 使用了自行設計的增強特徵的整流線性單位函數(changed ReLU, 簡稱cReLU),
	希望透過閥值過濾不重要的特徵,其公式如\cref{eq:eq-crelu}所示。
	作為閥值的 $c$ 值需要透過觀察資料集在高斯卷積模組的輸出來設定具體不同的$c$的數值來適應不同的資料集。
	然而一旦閥值設定過小則會削弱特徵增強的效果,
	閥值設定過大則會將大部分的特徵都歸零導致無法學習出有效特徵,
	因此如何設定 $c$ 則成為一個非常困難的問題,並且每次更換資料集都必須面對這個問題。

	為了解決閥值設定的問題,
	我們對cReLU的公式進行改善,
	首先設定一個希望保留的$RM$元素的百分比參數 $p\%$,RM的數目稱之為$C_{RM}$
	接著從同一影像在該層的所有 $RM$ 中取出第 $p\% * C_{RM}$  個最大的元素稱之為$P_{RM}$,
	並且將所有小於$P_{RM}$的數值歸零。
	如此一來,使用者只需要決定希望可以保留 $p\%$ 的數值,
	無須糾結於設定閥值$c$具體數值。
	這種方法簡化了參數設定過程,
	同時也確保了特徵增強的有效性。
	改善後的cReLU的公式為\\
	\begin{equation}
	    \label{eq:eq-cReLUPercent}
	    f(x)= 
	    \begin{cases}
	        0 & \text{if  $x < P_{RM}$ }\\
	        x & \text{if  $x \geq P_{RM}$}
	    \end{cases}, where P_{RM} = RM_{i}\left[ p\% * C_{RM} \right]
	\end{equation}


	\subsection{空間位置保留機制之優化設計}
	在空間保留機制模組上,
	CIM為了保留 $RM$ 之間的空間位置關係簡化了眼球跳動的方式,
	加入了一個可訓練的時間遺忘函數 $\alpha$ 進入合併公式\cref{eq:eq-sf}。
	CIM之合併公式:\\
		$RM_{c}$為合併後的RM,$RM_{k}$為第k張RM,$n$為輸入資訊的數量,$\alpha$為一個可訓練的參數
		\begin{equation}
		    \label{eq:eq-sf}
		    RM_{c}=\frac{1}{n} \sum_{k = 0}^{n-1} \alpha^{k} \times RM_{k}  \qquad ,\ where\ n = \textit{H}^{i}_{SF} \times \textit{W}^{i}_{SF}
		\end{equation}
	這樣會導致越早被看到(時間越早)的數值需要乘上的 $\alpha^{k}$ 的 $k$ 值越大,
	在合併時的加權也就越小。
	儘管這種做法確實可以使$RM$呈現出空間上的關係,
	然而當我們需要合併的$RM_{k}$越多時,
	隨著$k$ 值越來越大,$\alpha^{k}$會快速變小。
	這樣便造成當我們將$RM_{k}$ 乘上過小的的$\alpha^{k}$後,
	便會變得這個數值便會在$RM_{c}$中變得無足輕重。
	
	\colorbox {yellow}{可以放入$\alpha$的指數變化圖}

	對於大小較大的影像而言這種方法是有問題的,
	這會使得影像逐漸喪失一部分$ RM_{k} $ 所擷取的特徵。
	因此我們提出了將 $\alpha$ 的方法從$\alpha^{k}$,
	改成從[ $0.9 \sim 0.99$ ]之間進行等距採樣,
	得到 $k$ 個值代替 $\alpha^{k}$ 按照時序順序去乘以 $RM_{k}$ ,
	我們將第 $k$ 個RM所對應的 $\alpha$ 稱為 $\beta_{k}$,
	如此便可以改善當需要合併的 $RM$ 數量過多時 $\alpha^{k}$ 過小的問題,
	同時也保留了CIM當初設計時希望保留的 $RM$ 之間的對應空間關係。\\
	改善後的合併公式如\cref{eq:eq-sf-update},$\beta_{k}$從[$0.9 \sim 0.99$]之間取出的第$k$個值:\\
		\begin{equation}
		    \label{eq:eq-sf-update}
		    RM_{c}=\frac{1}{n} \sum_{k = 0}^{n-1} \beta_{k} \times RM_{k}  \qquad ,\ where\ n = \textit{H}^{i}_{SF} \times \textit{W}^{i}_{SF}
		\end{equation}
	RM的優化後的時序性合併如\cref{fig:SFN_update}。
	\fig[1][fig:SFN_update][H]{chapter3/SFM_update.png}[優化後空間位置保留機制示意圖][優化後空間位置保留機制]

	\subsection{模型流程的精簡}

	我們將原來在CIM的模型訓練流程中每層接在空間位置保留合併模組後的維度轉置與Reshape步驟(在\cite{YangCNNInterpretable}模型流程的 Step 4)去除,
	CIM會加入這個步驟的初衷在於方便之後輸出將濾波器輸出成可解釋性圖片,
	然而加入這個步驟導致一些不必要的問題:
	首先,CIM原本在第一層高斯卷積後的高斯卷積模組都只是做(1,1)的卷積操作,但是加入了這個步驟就會將後續的高斯卷積層的濾波器大小改變,容易引起閱讀者的誤解。
	其次,這個步驟並不會增加準確度或是增加效能,反而是多一些所轉置與Reshape所需要的時間。
	基於以上原因我們決定將此步驟捨棄,使過程容易理解、減少所需時間。

	\colorbox {yellow}{加入我們刪除了原先模型流程中的哪個步驟}


\pagebreak
\section{可解釋性} 
本研究使用了Yang 在2023年CIM論文中提出的特徵映射法做為解析模型的決策過程的基礎,
使得模型的決策過程可以被視覺化出來,
讓使用者了解模型關注的特徵資訊與決策背後的邏輯關係。
	
	\subsection{FM、RM、CI的意義}
	\label{section:InterablePicture}
	特徵映射圖(FM)、特徵映射響應圖(RM)、特徵映射圖之對應影像(CI),
	為Yang在CIM論文中提出的名詞,
	也是特徵映射法的核心三要素。

	特徵映射響應圖(RM)為當輸入進入高斯卷積模組後所得到的輸出,
	其代表的是輸入對於該高斯卷積模組的濾波器的相似度,
	也被視為輸入對該高斯卷機模組所有的濾波器的反應強度。
	同一層、不同位置的RM數值為輸入影像的不同位置對於某個濾波器的反應,
	不同層、同一位置的RM數值為輸入影像的同一位置對於不同濾波器的反應。

	特徵映射圖(FM)為將高斯卷積模組的濾波器提取出來並Reshape成二維矩陣所形成,
	由於模型在高斯卷積模組為計算輸入濾波器與相似度,
	因此FM可以被視為模型學習到了輸入影像中的何種特徵。

	特徵映射圖之對應影像(CI)為記錄資料集中所有影像對特定濾波器的反應,
	從中選出與該濾波器有最大的反應的影像,
	並該輸入影像視為該濾波器的對應影像。
	會需要CI的原因在於除了顏色感知層和輪廓感知層之外,
	後續的特徵傳遞層的輸入均為前一層所計算出來的相似度,
	人類已無法去解讀出FM的意義。
	因此我們需要透過CI來找出該濾波器與何種影像最相似,
	幫助使用者理解FM代表的特徵長相。

	RM\-CI為根據輸入影像在某層的高斯卷積模組的每一張RM的最大反應的位置來
	找到對應位置的CI影像。

	\subsection{色彩感知區塊之可解釋性}
	在色彩感知區塊的 FM 本身即可被視為 \textit{C}$_{out}$ 個不同RGB顏色,
	並且這個特性導致FM不會學習到輪廓的特徵。
	然而由於CI是由資料集中影像的部分區塊組成,
	因此FM對應的CI的影像反而卻同時存在顏色和輪廓,
	這會導致最後組合起來導致CI容易出現顏色相似,
	但輪廓與輸入影像完全不同的情況,看起來會十分怪異。
	為了解決這個問題,
	我們會取每個CI的平均色彩作為該CI的代表色,
	我們接著會將CI的代表色擴張為和原始CI相同大小色塊,
	並且使用這些色塊代替原本的CI進行呈現,
	從而更好的反應FM學習到的色彩特徵。

	總結色彩感知區塊產生可解釋性圖片的流程如下:
	\begin{itemize}
		\item [1]
		輸入影像後,找出色彩感應區塊的RM
		\item [2]
		從RM中找出每個RM的最大反應
		\item [3]
		根據RM的最大反應位置找出對應的CI
		\item [4]
		找出對應CI的代表色並擴張成與原始CI相同大小的色塊
		\item [5]
		利用色塊組出輸入影像在該層的代表影像
	\end{itemize}

	\subsection{色彩特徵傳遞區塊之可解釋性}
	在色彩特徵區塊上,
	由於經過空間合併模組的特徵合併,
	因此該區塊所學到的是比色彩感知區塊更加完整的特徵。
	為了準確地呈現FM對應的CI的不同的顏色與位置,
	我們採用以下的方法進行處理:\\
	首先,我們找到輸入影像在該區塊的高斯卷積模組對應的CI。
	接者,將這些CI切成與$CI_{color}$相同大小的小圖,
	然後,對這些小圖分別取平均顏色並將這些平均顏色擴張為和$CI_{color}$相同大小的色塊,
	用這些色塊代替原本的CI按照對應位置組成RM-CI影像。
	再加入了分割這個步驟後,每個CI都能反映出不同部分對應的顏色和位置,
	從而更好的呈現CI的顏色分布。

	總結色彩特徵傳遞區塊產生可解釋性圖片的流程如下:
	\begin{itemize}
		\item [1]
		輸入影像後,找出色彩感應區塊的RM
		\item [2]
		從RM中找出每個RM的最大反應
		\item [3]
		根據RM的最大反應位置找出對應的CI
		\item [4]
		將對應的CI分割成數個與$CI_{color}$相同大小的小圖
		\item [5]
		找出各個小圖的代表色並擴張成與原始CI相同大小的色塊
		\item [6]
		利用色塊組出輸入影像在該層的代表影像
	\end{itemize}

	\subsection{輪廓感知區塊和輪廓特徵傳遞區塊之可解釋性}
	輪廓感知和特徵傳遞區塊的可解釋性採用的是CIM的特徵圖對應法,
	利用輸入影像在每層的高斯卷積模組的RM,從RM中選出最大的反應強度,
	根據反應強度的位置找出對應的CI,並組出輸入影像在該層的代表影像。

	總結輪廓感知區塊和輪廓特徵傳遞區塊產生可解釋性圖片的流程如下:
	\begin{itemize}
		\item [1]
		輸入影像後,找出色彩感應區塊的RM
		\item [2]
		從RM中找出每個RM的最大反應
		\item [3]
		根據RM的最大反應位置找出對應的CI
		\item [4]
		利用色塊組出輸入影像在該層的代表影像
	\end{itemize}

\end{document}